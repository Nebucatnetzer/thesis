% Created 2018-07-09 Mo 20:55
% Intended LaTeX compiler: pdflatex
\documentclass[a4paper,11pt]{article}

%Enable URL highlighting
\usepackage{hyperref}
\usepackage{graphicx}
\usepackage[table,xcdraw]{xcolor}
\hypersetup{
    colorlinks=true,
    linkcolor=blue,
    urlcolor=blue,
}
\newcommand{\MYhref}[3][blue]{\href{#2}{\color{#1}{#3}}}%
\urlstyle{same}

%glossary
\usepackage[acronym,toc]{glossaries}
\renewcommand*{\acronymname}{Akronyme}
\makeglossaries
\setglossarystyle{altlistgroup}

\usepackage{xparse}
\DeclareDocumentCommand{\newdualentry}{ O{} O{} m m m m } {
  \newglossaryentry{gls-#3}{name={#5},text={#5\glsadd{#3}},
    description={#6},#1
  }
  \makeglossaries
  \newacronym[see={[Glossary:]{gls-#3}},#2]{#3}{#4}{#5\glsadd{gls-#3}}
}

\makeatletter
\renewcommand{\paragraph}{\@startsection{paragraph}{4}{0ex}%
    {-3.25ex plus -1ex minus -0.2ex}%
    {0.3ex plus 0.2ex}%
    {\normalfont\normalsize\bfseries}}
\makeatother

%page dimensions
\usepackage[paperwidth=216mm,paperheight=303mm,includeheadfoot, top=2.5cm, bottom=2.5cm, left=3cm, right=3cm]{geometry}
\geometry{bindingoffset=0.5cm}
\setlength{\marginparwidth}{0pt}

%Font settings
\usepackage{tgpagella}
\usepackage[T1]{fontenc}

%Support for German
\usepackage[utf8]{inputenc}
\usepackage[ngerman]{babel}

%number everything down to paragraphs
\setcounter{secnumdepth}{5}

%Enable Microtyping which improves justification
\usepackage{microtype}

%header
\setlength{\headheight}{15pt}
\usepackage{fancyhdr}
\pagestyle{fancy}
\lhead{\nouppercase{\leftmark}}
\rhead{}
\cfoot{\thepage}

%footnotes
\setlength{\skip\footins}{0.5cm}
\usepackage[bottom,hang]{footmisc}

%each section should start on a new page
\usepackage{titlesec}
\newcommand{\sectionbreak}{\clearpage}

%European style paragraphs
\setlength{\parskip}{\baselineskip}%
\setlength{\parindent}{0pt}%

%Make line spacing bigger
\linespread{1.3}

%Make whitespace in tables bigger
\renewcommand\arraystretch{1.5}

%Enable colours for tables
\usepackage[table]{xcolor}
\usepackage{longtable}

%Needed to rotate graphics and tables
\usepackage{rotating}
\usepackage{pdflscape}

%required for left alligned paragraph columns
\usepackage{array}

%Support for images
\usepackage{graphicx}
\usepackage{float}

%Material theme colours
\definecolor{red}{HTML}{F44336}
\definecolor{pink}{HTML}{E91E63}
\definecolor{purple}{HTML}{9C27B0}
%\definecolor{blue}{HTML}{2196F3}
\definecolor{brown}{HTML}{795548}
\definecolor{cyan}{HTML}{00BCD4}
\definecolor{darkgray}{HTML}{616161}
\definecolor{gray}{HTML}{9E9E9E}
\definecolor{lightgray}{HTML}{E0E0E0}
\definecolor{lime}{HTML}{CDDC39}
\definecolor{olive}{HTML}{827717}
\definecolor{orange}{HTML}{FF9800}
\definecolor{teal}{HTML}{009688}
\definecolor{yellow}{HTML}{FFEB3B}
\definecolor{green}{HTML}{388E3C}

%Support for Code Snippets and Syntax Highlighting
\usepackage{listings}
\usepackage{color}
\usepackage{courier}
\lstset{
  basicstyle=\ttfamily,
  language=bash,
  breaklines,
  literate={ö}{{\"o}}1
           {ä}{{\"a}}1
           {ü}{{\"u}}1,
  captionpos=b,                    % sets the caption-position to bottom
  commentstyle=\color{green},    % comment style
  escapeinside={\%*}{*)},          % if you want to add LaTeX within your code
  keywordstyle=\color{blue},       % keyword style
  stringstyle=\color{purple},     % string literal style
  showstringspaces=false,          % Removes the strange symboles where spaces are
}

% Listing with a box around it
\usepackage[most]{tcolorbox}
\AtBeginDocument{
\newtcblisting[blend into=listings]{sexylisting}[2][]{
    sharp corners,
    fonttitle=\bfseries,
    colframe=gray,
    listing only,
    listing options={basicstyle=\ttfamily,language=Python},
    title=#2, #1}
}
\renewcommand\lstlistingname{Code}
\renewcommand\lstlistlistingname{Code Ausschnitte}
\usepackage{tocbibind}
\usepackage[citestyle=verbose,bibstyle=numeric,sorting=none,backend=bibtex8]{biblatex}
\DefineBibliographyStrings{german}{%
  references = {Referenzen},
}
\renewcommand{\lstlistoflistings}{\begingroup
\tocfile{\lstlistlistingname}{lol}
\endgroup}

\renewcommand*{\glslinkcheckfirsthyperhook}{%
  \ifglsused{\glslabel}%
  {%
    \setkeys{glslink}{hyper=false}%
  }%
  {}%
}
\usepackage{emptypage}
\usepackage{afterpage}

\newcommand\blankpage{%
    \null
    \thispagestyle{empty}%
    \newpage}
\author{Andreas Zweili}
\date{\today}
\title{Projektantrag}
\hypersetup{
 pdfauthor={Andreas Zweili},
 pdftitle={Projektantrag},
 pdfkeywords={},
 pdfsubject={},
 pdfcreator={Emacs 25.2.2 (Org mode 9.1.13)},
 pdflang={Germanb}}
\begin{document}

\maketitle\newpage
\tableofcontents
\newpage

\section{Zweck des Dokuments}
\label{sec:org94b8dea}
Im Rahmen einer Diplomarbeit an den IBZ-Schulen soll die Grundlage für
ein neues Open Source Projekt geschaffen werden sowie die Entwicklung
der dazugehörigen Backup Applikation gestartet werden. Die Applikation
soll es normalen Usern ermöglichen mit der Backup Software BorgBackup
zu interagieren.

Das Dokument dient als Grundlage für die Freigabe der Diplomarbeit.

\section{Ausgangslage}
\label{sec:org42ce4b5}

Das Open Source Projekt \url{https://www.borgbackup.org/} entwickelt eine
Backup Software welche es einem ermöglicht Backups auf ein externes
Laufwerk oder einen externen Server zu machen. Für externe Server die
man nicht selber kontrolliert unterstützt BorgBackup auch eine starke
Verschlüsselung. Als weiteres wichtiges Feature unterstützt BorgBackup
auch Deduplikation was es einem erlaubt sehr viele Snapshots zu haben
ohne das der Speicher allzu fest wächst.

Als freie Software und Aufgrund der interessanten Features wäre
BorgBackup auch für weniger erfahrene User sehr nützlich. Leider gibt
es für BorgBackup zur Zeit kein klassisches grafisches User Interface
mit dem ein User interagieren kann. BorgBackup wird komplett über die
Kommando Zeile gesteuert.

\section{Projektrahmenbedingungen}
\label{sec:org8a07e68}

Das Projekt ist im Rahmen einer Diplomarbeit an den IBZ-Schulen zu
realisieren. Daraus ergeben sich Ansprüche an:

\begin{enumerate}
\item Komplexität des Vorhabens
\item Umfang des Vorhabens (250 Stunden)
\end{enumerate}

Die Begleitung der Diplomarbeit ist durch die IBZ sichergestellt.

Als nebenläufiges Ziel soll mit dieser Arbeit auch die Verbreitung von
freier Softare gefördert werden. Dies wird insbesondere dadurch
erreicht das die Software selbst unter der GNU Public License Version
3 veröffentlicht wird. Wenn möglich sollen während der Entwicklung
auch hauptsächlich freie Software verwendet werden. Die gesamte Arbeit
wird zudem zu jedem Zeitpunkt öffentlich einsehbar sein. Der Quelltext
der Dokumentation ist unter diesem Link erreichbar:
\url{https://git.2li.ch/Nebucatnetzer/thesis}

Das Repository für den Code der Applikation wird während der Umsetzung
erstellt. Die Dokumentation und der Code werden getrennt da das
Projekt auch nach der Abgabe weiter existieren soll.

\section{Zielsetzung}
\label{sec:org0692617}

\begin{table}[htbp]
\centering
\begin{tabular}{|p{9cm}|p{1.5cm}|p{2cm}|}
\hline
\textbf{Zielsetzung}\cellcolor[HTML]{C0C0C0} & \textbf{Muss}\cellcolor[HTML]{C0C0C0} & \textbf{Wunsch} (1-5, 5=sehr wichtig)\cellcolor[HTML]{C0C0C0}\\
\hline
 &  & \\
\hline
 &  & \\
\hline
 &  & \\
\hline
 &  & \\
\hline
\end{tabular}
\caption{\label{tab:orgba698b7}
Projektziele}

\end{table}

\section{Mittelbedarf}
\label{sec:orgc335aa3}

Für die Arbeit sind, ausser den 250h des Diplomanden, keine weiteren
Mittel notwendig.

\section{Planung}
\label{sec:orgb688b2c}

Die Arbeit ist innert 14 Wochen (Vorgabe IBZ-Schulen) abzuwickeln. Die
Verteilung der anstehenden Arbeiten auf die 14 Wochen ist durch den
Diplomanden vor dem Start der Phase Voranalyse zu erledigen.

\section{Wirtschaftlichkeit}
\label{sec:org18c6ec8}
Die Wirtschaftlichkeit ist für dieses Projekt nicht relevant.

\section{Konsequenzen}
\label{sec:org6bb20d7}
Keine speziellen Konsequenzen in dem Vorhaben bekannt.

\section{Antrag}
\label{sec:org3097edf}
Der Diplomand beantragt die Freigabe der Phase Voranalyse, diese
startet mit dem Kick-off vom 10.12.2018.
\end{document}