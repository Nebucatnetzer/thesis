\newglossaryentry{api}
{
    name={Application Programming Interface (API)},
    description={Ein Application Programming Interface (API), zu Deutsch
      Programmierschnittstelle, ist ein Programmteil, der von einem
      Softwaresystem anderen Programmen zur Anbindung an das System zur
      Verfügung gestellt wird},
    first={Programmierschnittstelle (API)},
    text={API}
}
\newglossaryentry{cc}
{
    name={Creative Commons (CC)},
    description={Die Creative Commons sind Lizenzen welche insbesondere für
      ``künstlerische'' Werke erstellt wurden. Also Texte, Videos und Fotos.
      Die Lizenz kann dabei sehr einfach an die Wünsche des Authors angepasst
      werden. Von sehr locker bis relativ restriktiv},
    first={Creative Commons (CC)},
    text={CC}
}
\newglossaryentry{bb}
{
    name={BorgBackup (Borg)},
    description={BorgBackup ist eine python basierte deduplizierende Backup
      Software. Welche sicher und effizient Backups erstellen kann},
    first={BorgBackup (kurz Borg)},
    text={Borg}
}
\newglossaryentry{dedup}
{
    name={Deduplikation},
    description={Blockbasierte Deduplikation vergleich Dateien auf der
      Block-Ebene und speichert anschliessend nur die zueinander
      unterschiedlichen Blöcke. Dadurch kann sehr viel Speicherplatz gespart
      werden wenn mehrere Versionen von grossen Dateien gesichert werden
      sollen},
    first={Deduplikation},
    text={Deduplikation}
}
\newglossaryentry{json}
{
    name={JavaScript Object Notation (JSON)},
    description={Die JavaScript Object Notation, kurz JSON, ist ein
      kompaktes Datenformat in einer einfach lesbaren Textform zum Zweck des
      Datenaustauschs zwischen Anwendungen.\footcite{json} Es wird vorallem in
      Webanwendungen häufig eingesetzt},
    first={JSON},
    text={JSON}
}
\newglossaryentry{libre}
{
    name={Freie Software},
    description={Freie Software (freiheitsgewährende Software, englisch free
      software oder auch libre software) bezeichnet Software, welche die
      Freiheit von Computernutzern in den Mittelpunkt stellt. Freie Software
      wird dadurch definiert, dass ein Nutzer mit dem Empfang der
      Software die Nutzungsrechte mitempfängt und diese ihm nicht vorenthalten
      oder beschränkt werden\footcite{libre}},
    first={freie Software},
    text={freie Software}
}
\newglossaryentry{bsd}
{
    name={Berkeley Software Distribution (BSD)},
    description={Die Berkeley Software Distribution war eine Software welche
      1977 entwickelt wurde. In abgeänderter Form existiert sie noch heute},
    first={ Berkeley Software Distribution (BSD)},
    text={BSD}
}
\newglossaryentry{gpl}
{
    name={GNU Public License (GPL)},
    description={Die GNU Public License ist eine der ältesten Lizenzen für
      freie Software. Die aktuellste Version ist GPLv3},
    first={GNU Public License (GPL)},
    text={GPL}
}