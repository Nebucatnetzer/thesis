%Enable URL highlighting
\usepackage{hyperref}
\usepackage{graphicx}
\usepackage[table,xcdraw]{xcolor}
\hypersetup{
    colorlinks=true,
    linkcolor=blue,
    urlcolor=blue,
}
\newcommand{\MYhref}[3][blue]{\href{#2}{\color{#1}{#3}}}%
\urlstyle{same}

%glossary
\usepackage[acronym,toc]{glossaries}
\renewcommand*{\acronymname}{Akronyme}
\makeglossaries
\setglossarystyle{altlistgroup}

\usepackage{xparse}
\DeclareDocumentCommand{\newdualentry}{ O{} O{} m m m m } {
  \newglossaryentry{gls-#3}{name={#5},text={#5\glsadd{#3}},
    description={#6},#1
  }
  \makeglossaries
  \newacronym[see={[Glossary:]{gls-#3}},#2]{#3}{#4}{#5\glsadd{gls-#3}}
}

\makeatletter
\renewcommand{\paragraph}{\@startsection{paragraph}{4}{0ex}%
    {-3.25ex plus -1ex minus -0.2ex}%
    {0.3ex plus 0.2ex}%
    {\normalfont\normalsize\bfseries}}
\makeatother

%page dimensions
\usepackage[paperwidth=216mm,paperheight=303mm,includeheadfoot, top=2.5cm, bottom=2.5cm, left=3cm, right=3cm]{geometry}
\geometry{bindingoffset=0.5cm}
\setlength{\marginparwidth}{0pt}

%Font settings
\usepackage{tgpagella}
\usepackage[T1]{fontenc}

%Support for German
\usepackage[utf8]{inputenc}
\usepackage[ngerman]{babel}

%number everything down to paragraphs
\setcounter{secnumdepth}{5}

%Enable Microtyping which improves justification
\usepackage{microtype}

%header
\setlength{\headheight}{15pt}
\usepackage{fancyhdr}
\pagestyle{fancy}
\lhead{\nouppercase{\leftmark}}
\rhead{}
\cfoot{\thepage}

%footnotes
\setlength{\skip\footins}{0.5cm}
\usepackage[bottom,hang]{footmisc}

%each section should start on a new page
\usepackage{titlesec}
\newcommand{\sectionbreak}{\clearpage}

%European style paragraphs
\setlength{\parskip}{\baselineskip}%
\setlength{\parindent}{0pt}%

%Make line spacing bigger
\linespread{1.3}

%Make whitespace in tables bigger
\renewcommand\arraystretch{1.5}

%Enable colours for tables
\usepackage[table]{xcolor}
\usepackage{longtable}

%Needed to rotate graphics and tables
\usepackage{rotating}
\usepackage{pdflscape}

%required for left alligned paragraph columns
\usepackage{array}

%Support for images
\usepackage{graphicx}
\usepackage{float}

%Material theme colours
\definecolor{red}{HTML}{F44336}
\definecolor{pink}{HTML}{E91E63}
\definecolor{purple}{HTML}{9C27B0}
%\definecolor{blue}{HTML}{2196F3}
\definecolor{brown}{HTML}{795548}
\definecolor{cyan}{HTML}{00BCD4}
\definecolor{darkgray}{HTML}{616161}
\definecolor{gray}{HTML}{9E9E9E}
\definecolor{lightgray}{HTML}{E0E0E0}
\definecolor{lime}{HTML}{CDDC39}
\definecolor{olive}{HTML}{827717}
\definecolor{orange}{HTML}{FF9800}
\definecolor{teal}{HTML}{009688}
\definecolor{yellow}{HTML}{FFEB3B}
\definecolor{green}{HTML}{388E3C}

%Support for Code Snippets and Syntax Highlighting
\usepackage{listings}
\usepackage{color}
\usepackage{courier}
\lstset{
  basicstyle=\ttfamily,
  language=bash,
  breaklines,
  literate={ö}{{\"o}}1
           {ä}{{\"a}}1
           {ü}{{\"u}}1,
  captionpos=b,                    % sets the caption-position to bottom
  commentstyle=\color{green},    % comment style
  escapeinside={\%*}{*)},          % if you want to add LaTeX within your code
  keywordstyle=\color{blue},       % keyword style
  stringstyle=\color{purple},     % string literal style
  showstringspaces=false,          % Removes the strange symboles where spaces are
}

% Listing with a box around it
\usepackage[most]{tcolorbox}
\AtBeginDocument{
\newtcblisting[blend into=listings]{sexylisting}[2][]{
    sharp corners,
    fonttitle=\bfseries,
    colframe=gray,
    listing only,
    listing options={basicstyle=\ttfamily,language=Python},
    title=#2, #1}
}
\renewcommand\lstlistingname{Code}
\renewcommand\lstlistlistingname{Code Ausschnitte}
\usepackage{tocbibind}
\usepackage[citestyle=verbose,bibstyle=numeric,sorting=none,backend=bibtex8]{biblatex}
\DefineBibliographyStrings{german}{%
  references = {Referenzen},
}
\renewcommand{\lstlistoflistings}{\begingroup
\tocfile{\lstlistlistingname}{lol}
\endgroup}

\renewcommand*{\glslinkcheckfirsthyperhook}{%
  \ifglsused{\glslabel}%
  {%
    \setkeys{glslink}{hyper=false}%
  }%
  {}%
}
\usepackage{emptypage}
\usepackage{afterpage}

\newcommand\blankpage{%
    \null
    \thispagestyle{empty}%
    \newpage}